\documentclass[12pt]{report} %fuente a 12pt

% MÁRGENES: 2,5 cm sup. e inf.; 3 cm izdo. y dcho.
\usepackage[
a4paper,
vmargin=2.5cm,
hmargin=3cm
]{geometry}

% ENLACES
\usepackage{hyperref}
\hypersetup{colorlinks=true,
	linkcolor=black, % enlaces a partes del documento (p.e. índice) en color negro
    urlcolor=blue} % enlaces a recursos fuera del documento en azul
    
\usepackage[spanish, es-tabla]{babel} 

\title{Lyncex: describiendo una aplicación web como conocimiento}
\author{Adrián Arroyo Calle}
\date{Curso 2019-2020}

\begin{document}

\maketitle

\section{¿Qué es?}
Lyncex es una base de datos y a la vez un servidor web configurable a través del contenido semántico de la propia base de datos.

\section{Introducción}
En ciencias de la computación, de forma recurrente se divide entre código, lo que va a ejecutar la máquina, y datos.
Esta diferencia, aunque pueda resultar evidente, es innecesaria, ya que el código no deja de ser dato, solo que con una semántica diferente.
Von Neuman, en su modelo de computadora, elimina las diferencias a nivel de hardware entre código y datos, de modo muy exitoso, hasta tal punto que esta idea sigue siendo la base de los procesadores modernos actuales.
Hoy día en la creación de aplicaciones web, separamos por un lado el código y por otro los datos que van a circular a través de él. 
No obstante, considero interesante imaginar y plantear una aplicación web descrita de la misma forma en que se describen los datos.
De forma principalmente declarativa y usando tecnologías maduras como RDF como la base del modelo de datos.
El servidor web pasa a ser una base de datos, dónde las diferencias entre código y datos son puramente semánticas.

\section{Estado del Arte}

\bibliographystyle{plain}

\bibliography{ref}

\end{document}